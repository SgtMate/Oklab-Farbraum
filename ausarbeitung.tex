\documentclass[12pt, a4paper, ngerman]{article}

% Metadata Setup
\newcommand{\Autor}{Leon Kampwerth}
\newcommand{\Was}{Hausarbeit Digitale Bildverarbeitung}
\newcommand{\Kurs}{TINF20IN}
\newcommand{\MatrikelNummer}{5722356}
\newcommand{\Studiengang}{Digitale Bildverarbeitung}

\title{Perceptually Uniform Color Spaces und der Oklab Farbraum}
\author{\Autor}
\date{22.03.2023}

% SETUP
\usepackage{biblatex} % für bibliografie
\usepackage{hyperref} % für links zum klicken
\usepackage{color}    % für Farben (benötigt für listings)
\usepackage{listings} % code schnipsel
\usepackage[ngerman]{babel} % lokalisierung der Titel (Inhaltsverzeichniss)
\usepackage{bookmark} % bookmarks für das PDF
\usepackage{csquotes} % korrekte quotes
\usepackage[version=3]{acro} % akronyme
\usepackage{geometry} % seitengeometrie (margin etc einstellen)
\usepackage{parskip}  % zeilenabstand bei neuem paragraph statt indentierung
\usepackage{fancyhdr} % header und footer
\usepackage{array}    % für bessere Tabellen
\usepackage{titlesec} % um die Titel anzupassen
\usepackage{plantuml} % PLANTUML_JAR has to be set and --shell-escape
\usepackage{amsfonts} % für \mathbb
\usepackage{placeins} % für \FloatBarrier
\usepackage{nicematrix}
 
\hypersetup{
  pdfauthor={\Autor},
  pdftitle={\Was},
  hidelinks
}

\geometry{
  a4paper,
  left=25mm,
  right=25mm,
  headheight=125mm,
  top=35mm,
  bottom=30mm,
  footskip=15mm
}

% title setup 
% make paragraph have a newline
\titleformat{\paragraph}
{\normalfont\normalsize\bfseries}{\theparagraph}{1em}{}
\titlespacing*{\paragraph}
{0pt}{3.25ex plus 1ex minus .2ex}{1.5ex plus .2ex}

% add bibliography
\addbibresource{bibliography.bib}

% header and footer setup
\pagestyle{fancy}
\fancyhf{}
\rhead{\Was}
\lhead{\leftmark}
\lfoot{Autor: \Autor, Kurs: \Kurs}
\rfoot{Seite \thepage}
\renewcommand{\headrulewidth}{1pt}
\renewcommand{\footrulewidth}{1pt}
\fancypagestyle{simple}{
  \fancyhf{}
  \rhead{\Was}
  \lfoot{Autor: \Autor, Kurs: \Kurs}
  \rfoot{Seite \thepage}
}

% acronyms
\acsetup{
  list/display = used,
  pages/display = first
}

%Offizele Akronyme

\newcommand{\reals}{\ensuremath{\mathbb{R}}}
\newcommand{\natnums}{\ensuremath{\mathbb{N}}}

% code snippet setup
\renewcommand{\lstlistingname}{Code-Auszug}
\renewcommand{\lstlistlistingname}{Liste der Code-Auszüge}

\definecolor{black}{rgb}{0,0,0}
\definecolor{green}{rgb}{0,0.5,0}
\definecolor{orange}{rgb}{1,0.45,0.13}		
\definecolor{brown}{rgb}{0.69,0.31,0.31}

% python
\lstdefinelanguage{Python}{
  morekeywords={import, def, from, for, in, if, else, return, True, False, catch, return, null, switch, if, in, while, do, else, case, break},
  morecomment=[l]\#,
  morestring=[b]",
  morestring=[b]""",
  morestring=[b]'
}

\lstdefinestyle{light}{
  % General design
  basicstyle={\footnotesize\ttfamily},   
  frame=b,
  % line-numbers
  xleftmargin={0.75cm},
  numbers=left,
  stepnumber=1,
  firstnumber=1,
  numberfirstline=true,	
  % Quellcode design
  identifierstyle=\color{black},
  keywordstyle=\color{blue}\bfseries,
  ndkeywordstyle=\color{green}\bfseries,
  stringstyle=\color{orange}\ttfamily,
  commentstyle=\color{brown}\ttfamily,
  % Quellcode
  alsodigit={.:;},
  tabsize=2,
  showtabs=false,
  showspaces=false,
  showstringspaces=false,
  extendedchars=true,
  breaklines=true,
}

\begin{document}
\raggedright % sorgt dafür das alles strikt links ausgerichtet wird (und sorgt für mehr seiten)


% Titlepage
\makeatletter
\begin{titlepage}
  \begin{center}
    \vspace*{1cm}
    {\Huge\scshape \Was}\\[2cm]
    \begin{center}
      \linespread{1}\Huge \@title\\[2cm]
    \end{center}
    {\large \Studiengang}\\
    {\large Duale Hochschule Baden-Württemberg\\ Stuttgart}\\[2cm]
    {\large von}\\
    {\large\bfseries \@author}
    \vfill
  \end{center}
  \begin{tabular}{l@{\hspace{2cm}}l}
    Matrikelnummer: & \MatrikelNummer \\
    Abgabedatum:    & \@date          \\
  \end{tabular}
\end{titlepage}
\makeatother

% Table of content
\tableofcontents
\newpage

\thispagestyle{simple}
\printacronyms[name=Abkürzungsverzeichnis, heading=section*]
\newpage

%%%%%%
% Content here
%%%%%% 

\renewcommand{\abstractname}{Abstract} % dass Abstract auch Abstract heißt und nicht zusammenfassung
\begin{abstract}
Dies ist ein Beispiel für ein Abstract \cite{Mustermann:2023}.
\end{abstract}

\section{Farben und Farbräume}
Was ist Farbe? Das ist eine Frage die sehr einfach zu Beantworten scheint, da fast jeder Farben sehen kann. 
Die Encyclopedia Britannica definiert Farbe sinngemäß als Eigenschaft eines Objektes, die durch dessen Farbton, 
Helligkeit und Sättigung beschrieben werden kann. In der Physik werden Farben mit Elektromagnetischer Strahlung in einem
bestimmten Bereich des elektromagnetischen Spektrums beschrieben, der für das menschliche Auge sichtbar ist~\cite{Nassau_2023}.
Gerade hier liegt ein Problem vor, da Farben sowohl Phasikalisch erklärt werden können, 
aber auch Teil der Menschlichen wahrnehmeung sind. Und diese beiden Sichtweisen sind nicht immer deckungsgleich.

\paragraph{Farbwahrnehmung durch das menschliche Auge}
Die Farbwahrnehmung des Menschen besteht aus mehreren Schritten, 
welche von dem Einfallen der Lichtstrahlen in das Auge bis zur Interpretation der Farbe durch das Gehirn reichen.
Im Auge fällt das Licht auf die Netzhaut, wo sich Zapfen- und Stäbchenzellen befinden, 
welche Photorezeptoren sind, die das Licht in Signale für das Hirn umwandeln.
Für die Farbwahrnehmung sind die Zapfen zuständig, von denen es drei Arten gibt, welche auf unterschieliche Wellenlängen reagieren.
S-Zapfen reagieren auf Wellenlängen im blauen Bereich des sichtbaren Spetrums (ca. 420nm), 
M-Zapfen auf Wellenlängen im grünen Bereich (ca. 530nm) und L-Zapfen auf Wellenlängen im gelb-grünen Bereich (ca. 560nm).
Auch wenn der L-Zapfen auf Licht im gelb-grünen Bereich am stärksten reagiert, 
ist er am wichtigsten für die Wahrnehmung von Rot und wird daher auch als Rotrezeptor bezeichnet~\cite{Zapfen_Auge_2023}.
Die Farbwahrnehmung des Menschen entsteht durch das zusammenspiel der drei Zapfen, 
die, wie in Grafik~\ref{fig:LMS} zu erkennen, durch die Wellenlängen unterschiedlich stark angeregt werden. 

\begin{figure}
  \centering
  \includegraphics[width=0.5\textwidth]{Grafiken/LMS.png}
  \caption{Normalisierte Empfindlichkeitsspektren menschlicher Zapfenzellen. X Achse: Wellenlänge in nm, Y Achse: Empfindlichkeit, anteilig. Quelle: \cite{LMS_color_space_2023}}
  \label{fig:LMS}
\end{figure}

\paragraph{Farbinterpretation durch das Gehirn}
Die Signale der Zapfen werden im Hirn verarbeitet und interpretiert. 
Dabei werden die Farben durch den Mensch in Form unterschiedlicher Farbeigenschaften wahrgenommen.
Zu diesen Farbeigenschaften zählen der Farbton, die Leuchtkraft, die Helligkeit, das Chroma und die Sättigung.
Die Bedeutung dieser Begriffe wird im später Verlaufe des Artikels noch näher beleuchtet.
Wichtig ist hier, dass die Wahrnehmung dieser Eigenschaften durch viele psychologische Phenomene beeinflusst wird.
Die chromatische Anpassung beschreibt zum Beispiel die Fähigkeit der menschlichen Farbwahrnehmung, 
bei der Betrachtung eines reflektierenden Objekts vom Weißpunkt der beleuchtenden Lichtquelle zu abstrahieren. 
Für das menschliche Auge sieht ein weißes Blatt Papier weiß aus, egal ob die Beleuchtung bläulich oder gelblich ist.
Weitere solcher effekte sind der Bezold-Brücke Effekt und der Abney Effekt, welche sich auf die wahrnehmeung des Farbtons auswirken,
der Stevens Effekt, welcher sich auf den Kontrast auswirkt und der Helmholtz-Kohlrausch Effekt, der sich auf die Leuchtkraft auswirkt.
All diese Effekte sorgen dafür das die Farbwahrnehmung des Menschen schwer zu beschreiben ist und Modelle, die dies Versuchen sehr komplex werden können~\cite{Color_appearance_model_2023}.

\subsection{Farbräume}


\subsection{Begriffsdefinitionen im Bereich der Farbräume}
\subsection{Unterschieliche Farbräume}
%hier eine auswahl wichtiger farbräume kurz abhandeln, am besten die farbräume die thematisiert werden

\section{Der Oklab Farbraum}
\subsection{Das Problem mit herkömmlichen Farbräumen}
\subsection{Motivation für Oklab}
\paragraph{Helmholtz-Kohlrausch Effekt}
\subsection{Herleitung von Oklab}
\subsection{Wie gut ist Oklab?}
\subsection{Bessere Sättigungskorrektur mittels Oklab}
% Von dem anderen Artikel
\subsection{Oklab in der Praxis}

% Bibliography
\printbibliography

\end{document}
