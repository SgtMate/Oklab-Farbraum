\documentclass[12pt, a4paper, ngerman]{article}

% Metadata Setup
\newcommand{\Autor}{Leon Kampwerth}
\newcommand{\Was}{Hausarbeit Digitale Bildverarbeitung}
\newcommand{\Kurs}{TINF20IN}
\newcommand{\MatrikelNummer}{5722356}
\newcommand{\Studiengang}{Digitale Bildverarbeitung}

\title{Perceptually Uniform Color Spaces und der Oklab Farbraum}
\author{\Autor}
\date{22.03.2023}

% SETUP
\usepackage{biblatex} % für bibliografie
\usepackage{hyperref} % für links zum klicken
\usepackage{color}    % für Farben (benötigt für listings)
\usepackage{listings} % code schnipsel
\usepackage[ngerman]{babel} % lokalisierung der Titel (Inhaltsverzeichniss)
\usepackage{bookmark} % bookmarks für das PDF
\usepackage{csquotes} % korrekte quotes
\usepackage[version=3]{acro} % akronyme
\usepackage{geometry} % seitengeometrie (margin etc einstellen)
\usepackage{parskip}  % zeilenabstand bei neuem paragraph statt indentierung
\usepackage{fancyhdr} % header und footer
\usepackage{array}    % für bessere Tabellen
\usepackage{titlesec} % um die Titel anzupassen
\usepackage{plantuml} % PLANTUML_JAR has to be set and --shell-escape
\usepackage{amsfonts} % für \mathbb
\usepackage{placeins} % für \FloatBarrier
\usepackage{nicematrix}
 
\hypersetup{
  pdfauthor={\Autor},
  pdftitle={\Was},
  hidelinks
}

\geometry{
  a4paper,
  left=25mm,
  right=25mm,
  headheight=125mm,
  top=35mm,
  bottom=30mm,
  footskip=15mm
}

% title setup 
% make paragraph have a newline
\titleformat{\paragraph}
{\normalfont\normalsize\bfseries}{\theparagraph}{1em}{}
\titlespacing*{\paragraph}
{0pt}{3.25ex plus 1ex minus .2ex}{1.5ex plus .2ex}

% add bibliography
\addbibresource{bibliography.bib}

% header and footer setup
\pagestyle{fancy}
\fancyhf{}
\rhead{\Was}
\lhead{\leftmark}
\lfoot{Autor: \Autor, Kurs: \Kurs}
\rfoot{Seite \thepage}
\renewcommand{\headrulewidth}{1pt}
\renewcommand{\footrulewidth}{1pt}
\fancypagestyle{simple}{
  \fancyhf{}
  \rhead{\Was}
  \lfoot{Autor: \Autor, Kurs: \Kurs}
  \rfoot{Seite \thepage}
}

% acronyms
\acsetup{
  list/display = used,
  pages/display = first
}

%Offizele Akronyme

\newcommand{\reals}{\ensuremath{\mathbb{R}}}
\newcommand{\natnums}{\ensuremath{\mathbb{N}}}

% code snippet setup
\renewcommand{\lstlistingname}{Code-Auszug}
\renewcommand{\lstlistlistingname}{Liste der Code-Auszüge}

\definecolor{black}{rgb}{0,0,0}
\definecolor{green}{rgb}{0,0.5,0}
\definecolor{orange}{rgb}{1,0.45,0.13}		
\definecolor{brown}{rgb}{0.69,0.31,0.31}

% python
\lstdefinelanguage{Python}{
  morekeywords={import, def, from, for, in, if, else, return, True, False, catch, return, null, switch, if, in, while, do, else, case, break},
  morecomment=[l]\#,
  morestring=[b]",
  morestring=[b]""",
  morestring=[b]'
}

\lstdefinestyle{light}{
  % General design
  basicstyle={\footnotesize\ttfamily},   
  frame=b,
  % line-numbers
  xleftmargin={0.75cm},
  numbers=left,
  stepnumber=1,
  firstnumber=1,
  numberfirstline=true,	
  % Quellcode design
  identifierstyle=\color{black},
  keywordstyle=\color{blue}\bfseries,
  ndkeywordstyle=\color{green}\bfseries,
  stringstyle=\color{orange}\ttfamily,
  commentstyle=\color{brown}\ttfamily,
  % Quellcode
  alsodigit={.:;},
  tabsize=2,
  showtabs=false,
  showspaces=false,
  showstringspaces=false,
  extendedchars=true,
  breaklines=true,
}

\begin{document}
\raggedright % sorgt dafür das alles strikt links ausgerichtet wird (und sorgt für mehr seiten)


% Titlepage
\makeatletter
\begin{titlepage}
  \begin{center}
    \vspace*{1cm}
    {\Huge\scshape \Was}\\[2cm]
    \begin{center}
      \linespread{1}\Huge \@title\\[2cm]
    \end{center}
    {\large \Studiengang}\\
    {\large Duale Hochschule Baden-Württemberg\\ Stuttgart}\\[2cm]
    {\large von}\\
    {\large\bfseries \@author}
    \vfill
  \end{center}
  \begin{tabular}{l@{\hspace{2cm}}l}
    Matrikelnummer: & \MatrikelNummer \\
    Abgabedatum:    & \@date          \\
  \end{tabular}
\end{titlepage}
\makeatother

% Table of content
\tableofcontents
\newpage

\thispagestyle{simple}
\printacronyms[name=Abkürzungsverzeichnis, heading=section*]
\newpage

%%%%%%
% Content here
%%%%%% 

\section{Farben und Farbräume}
Was ist Farbe? Das ist eine Frage die sehr einfach zu Beantworten scheint, da fast jeder Farben sehen kann. 
Die Encyclopedia Britannica definiert Farbe sinngemäß als Eigenschaft eines Objektes, die durch dessen Farbton, 
Helligkeit und Sättigung beschrieben werden kann. In der Physik werden Farben mit Elektromagnetischer Strahlung in einem
bestimmten Bereich des elektromagnetischen Spektrums beschrieben, der für das menschliche Auge sichtbar ist~\cite{Nassau_2023}.
Gerade hier liegt ein Problem vor, da Farben sowohl Phasikalisch erklärt werden können, 
aber auch Teil der Menschlichen wahrnehmeung sind.

\paragraph{Farbwahrnehmung durch das menschliche Auge}
Die Farbwahrnehmung des Menschen besteht aus mehreren Schritten, 
welche von dem Einfallen der Lichtstrahlen in das Auge bis zur Interpretation der Farbe durch das Gehirn reichen.
Im Auge fällt das Licht auf die Netzhaut, wo sich Zapfen- und Stäbchenzellen befinden, 
welche Photorezeptoren sind, die das Licht in Signale für das Hirn umwandeln.
Für die Farbwahrnehmung sind die Zapfen zuständig, von denen es drei Arten gibt, welche auf unterschieliche Wellenlängen reagieren.
S-Zapfen reagieren auf Wellenlängen im blauen Bereich des sichtbaren Spetrums (ca. 420nm), 
M-Zapfen auf Wellenlängen im grünen Bereich (ca. 530nm) und L-Zapfen auf Wellenlängen im gelb-grünen Bereich (ca. 560nm).
Auch wenn der L-Zapfen auf Licht im gelb-grünen Bereich am stärksten reagiert, 
ist er am wichtigsten für die Wahrnehmung von Rot und wird daher auch als Rotrezeptor bezeichnet~\cite{Zapfen_Auge_2023}.
Die Farbwahrnehmung des Menschen entsteht durch das zusammenspiel der drei Zapfen, 
die, wie in Grafik~\ref{fig:LMS} zu erkennen, durch die Wellenlängen unterschiedlich stark angeregt werden. 
Dies kann auch dafür sorgen, dass unterschiedliche Lichtspektren als dieselbe Farbe wahrgenommen werden, 
da sie die unterschiedlichen Zapfen in gleicher Weise erregen. 
So ist es möglich, dass Displays nur durch die Kombination der drei Grundfarben Rot, Grün und Blau 
so viele Farben darstellen können ~\cite{Ottosson_2020}.

\begin{figure}
  \centering
  \includegraphics[width=0.5\textwidth]{Grafiken/LMS.png}
  \caption{Normalisierte Empfindlichkeitsspektren menschlicher Zapfenzellen. X Achse: Wellenlänge in nm, Y Achse: Empfindlichkeit, anteilig. Quelle: \cite{LMS_color_space_2023}}
  \label{fig:LMS}
\end{figure}

\paragraph{Farbinterpretation durch das Gehirn}
Die Signale der Zapfen werden im Hirn verarbeitet und interpretiert. 
Dabei werden die Farben durch den Mensch in Form unterschiedlicher Farbeigenschaften wahrgenommen, 
die nicht als physikalische Eigenschaften der Farbe definiert werden können~\cite{Ottosson_2020}.
Zu diesen Farbeigenschaften zählen der Farbton, die Leuchtkraft, die Helligkeit, das Chroma und die Sättigung.
Wichtig ist hier, dass die Wahrnehmung dieser Eigenschaften durch viele psychologische Phenomene beeinflusst wird.
Die chromatische Anpassung beschreibt zum Beispiel die Fähigkeit der menschlichen Farbwahrnehmung, 
bei der Betrachtung eines reflektierenden Objekts vom Weißpunkt der beleuchtenden Lichtquelle zu abstrahieren. 
Für das menschliche Auge sieht ein weißes Blatt Papier weiß aus, egal ob die Beleuchtung bläulich oder gelblich ist.
Weitere solcher effekte sind der Bezold-Brücke Effekt und der Abney Effekt, welche sich auf die wahrnehmeung des Farbtons auswirken,
der Stevens Effekt, welcher sich auf den Kontrast auswirkt und der Helmholtz-Kohlrausch Effekt, der sich auf die Leuchtkraft auswirkt.
All diese Effekte sorgen dafür das die Farbwahrnehmung des Menschen schwer zu beschreiben ist und Modelle, die dies Versuchen sehr komplex werden können~\cite{Color_appearance_model_2023}.

\subsection{Farbräume}
Ein Farbsystem stellt das Grundprinzip einer Farbmischung dar, 
beispielsweise durch das Mischen der Lichtfarben Rot, Grün und Blau oder durch das Mischen von Farbpigmenten.
Farbmodelle werden aus einem solchen Farbsystem abgeleitet und ordnen Farben einen eindeutigen Wert (Farborte) zu.
Farbmodelle sind häufig dreidimensional, damit sie einfach visualisiert werden können.
Ein Farbraum wiederum beschreibt alle Farben eines Farbmodells, die durch farbgebende Methoden tatsächlich ausgegeben werden können~\cite{Farbraum_2023}.
Ein Beispiel ist der sRGB Farbraum, welcher ursprünglich für CRTs entwickelt wurde, aber auch heute noch von vielen Displays verwendet wird~\cite{sRGB-Farbraum_2019}.
Die darstellbaren Farben bilden innerhalb des Farbmodells einen Körper, der als Gamut bezeichnet wird.

\subsection{Farbeigenschaften} 
Abgesehen von den von RGB bekannten Farbdimensionen, welche den Anteil der Grundfarben Rot, Grün und Blau beschreiben, 
gibt es noch weitere mögliche Farbdimensionen. Diese Dimensionen nehmen Bezug auf die Farbeigenschaften, nach denen die Farben beschrieben werden können.
Diese Farbeigenschaften stehen im Bezug zueinander und beeinflussen sich teilweise gegenseitig~\ref{fig:ExColordimension}.

\paragraph{Farbton}
Der Farbton (eng. hue) ist eine Eigenschaft der visuellen Wahrnehmung, in welcher eine Fläche 
ähnlich zu einer der Farben rot, gelb, grün oder blau erscheint oder eine Kombination von 
aneinanderliegenden Paaren dieser Farben in einem geschlossenen Farbring ähnelt (vgl. Grafik \ref{fig:Hue})~\cite{Darktable_2023}. 

\begin{figure}
  \centering
  \includegraphics[width=0.5\textwidth]{Grafiken/Farbring.png}
  \caption{Beispiel eines Farbrings. Quelle: \cite{Hue_2023}}
  \label{fig:Hue}
\end{figure}

\paragraph{Leuchtkraft (und Brillanz)}
Leuchtkraft (eng. brightness) ist eine Eigenschaft der visuellen Wahrnehmung, 
nach der eine Fläche mehr oder weniger Licht zu emmitieren oder zu reflektieren scheint. 
Die Brillanz (eng. brilliance) ist die Leuchtkraft einer Fläche relativ betrachtet zu ihrer Umgebung.
Leuchtkraft ist ein absoluter Wert, während Brillanz in Relation zur Umgebung steht.
Da bei der Bildverarbeitung eine Erhöhung der Leuchtkraft auch die Brillanz erhöht, 
werden diese Begriffe häufig synonym verwendet~\cite{Darktable_2023}.

\paragraph{Helligkeit}
Die Helligkeit (eng. lightness) einer Fläche beschreibt die Leuchtkraft der Fläche 
relativ zu einer ähnlich beleuteten weißen oder stark reflektierenden Fläche~\cite{Darktable_2023}.

\paragraph{Chroma (Buntheit)}
Chroma beschreibt die Farbigkeit einer Fläche, beurteilt im verhältnis zur Leuchtkraft 
einer ähnlich beleuchteten grauen oder weißen Fläche~\cite{Darktable_2023}. 

\paragraph{Sättigung}
Die Sättigung (eng. saturation) ist die Farbigkeit einer Fläche im verhältnis zu ihrer Leuchtkraft~\cite{Darktable_2023}.

\begin{figure}
  \centering
  \includegraphics[width=\textwidth]{Grafiken/KoordinatenFarbeigenschaften.png}
  \caption{Beispiel für den Zusammenhang zwischen den Farbeigenschaften. Quelle: \cite{Darktable_2023}}
  \label{fig:ExColordimension}
\end{figure}

\subsection{Unterschieliche Farbräume}
Farben können in vielen unterschiedlichen Farbräumen dargestellt werden. 
Dabei braucht jede Farbe unabhängig vom Farbraum mindestens drei Werte, um beschrieben werden zu können.
Zum einen eine Metrik für Helligkeit oder Leuchtkraft und zum anderen zwei Metriken für die Chromazität 
(nicht zu verwechseln mit Chroma). 
Diese zwei Metriken können zum Beispiel Farbton und Buntheit/Chroma oder komplementäre Farbkoordinaten sein~\cite{Darktable_2023}.
Im Folgenden werden einige Farbräume kurz vorgestellt. Diese Farbräume werden im weiteren Verlauf des Artikels nochmal aufgegriffen.

\paragraph{HSV und HSL}
HSV steht für Hue, Saturation und Value und HSL für Hue, Saturation und Lightness.
Beide Farbräume sind alternative Darstellungen des RGB Farbraums. 
HSL und HSV können beide zylindrische Farbkörper (Abbildung \ref{fig:HSV_HSL}) dargestellt werden, 
wobei hue den Winkel und Saturation den Radius definieren.
Die Höhe des Zylinders ist der Wert, der die Helligkeit als Lightness oder Value beschreibt.
HSV oder HSL werden in Endnutzeranwendungen häufig für Color Picker verwendet ~\cite{HSL_and_HSV_2023}.

\begin{figure}
  \centering
  \includegraphics[width=0.48\textwidth]{Grafiken/HSV_Zylinder.png}
  \includegraphics[width=0.48\textwidth]{Grafiken/HSL_Zylinder.png}
  \caption{Beispiel von Farbkörpern für HSV links und HSL rechts. Quelle: ~\cite{HSL_and_HSV_2023}}
  \label{fig:HSV_HSL}
\end{figure}

\paragraph{CIE 1931 XYZ}
Die CIE-Farbräume von 1931 sind die ersten definierten quantitativen Verbindungen zwischen den Verteilungen 
der Wellenlängen im elektromagnetischen sichtbaren Spektrum und den physiologisch wahrgenommenen Farben im menschlichen Farbsehen. 
Bei CIE XYZ werden die drei Farbkoordinaten X, Y und Z verwendet, um die Farbe zu beschreiben.
Dabei ist Y die Lumineszenz, Z ist ähnlich zu Blau im RGB Farbraum und X ist eine Mischung der drei CIE RGB Kurven.
Vorteil ist hier, dass bei einem konstanten Y Wert die XZ Fläche alle Farben dieser Lumineszenz enthält~\cite{CIE_1931_color_space_2023}.

\paragraph{CIELAB}
Der CIELAB Farbraum wurde 1976 von der CIE entwickelt.
CIELAB hat drei dinemsionen und erstreckt sich über den gesamten Gamut des menschlichen Farbsehens (siehe Grafik \ref{fig:CIELAB}).
Die drei Farbkoordinaten orientieren sich ebenfalls an der menschlichen Wahrnehmung, 
wo Rot und Grün, sowie Blau und Gelb als komplementäre Farben wahrgenommen werden.
Um dies aufzugreifen werden die Farbkoordinaten L, a und b verwendet.
Die L-Achse ist die Helligkeit und definiert 0 als Schwarz und 100 als Weiß.
Die a-Achse ist eine Relative Achse für die Farbtonkomplementarität von Rot und Grün.
Negative Werte gehen in Richtung Grün, positive Werte in Richtung Rot.
Die b-Achse ist eine Relative Achse für die Farbtonkomplementarität von Blau und Gelb.
Negative Werte gehen in Richtung Blau, positive Werte in Richtung Gelb.
CIELAB wird immer in relation zu einem Referenzweiß definiert, 
wofür CIE die Nutzung von CIE Standard illuminant D65 empfiehlt ~\cite{CIELAB_color_space_2023}.

\begin{figure}
  \centering
  \includegraphics[width=0.29\textwidth]{Grafiken/CIELAB1.png}
  \includegraphics[width=0.29\textwidth]{Grafiken/CIELAB2.png}
  \includegraphics[width=0.29\textwidth]{Grafiken/CIELAB3.png}
  \caption{a*b* Farbebene des CIELAB Farbraumes bei unterschiedlichen L* Werten. Quelle: ~\cite{CIELAB_color_space_2023}}
  \label{fig:CIELAB}
\end{figure}

\paragraph{CAM16-UCS}
In der Farbmetrik ist der OSA-UCS (Optical Society of America Uniform Color Space) ein Farbraum, 
der erstmals 1947 veröffentlicht und vom Committee on Uniform Color Scales der Optical Society of America entwickelt wurde.
Der Ausschuss beschloss, dass eine neue Form verwendet werden muss, 
um einheitliche Farbunterschiede in jeder Richtung genau darstellen zu können.
Die drei dimensionen des Farbraumes sind sehr ähnlich zu CIELAB.
Die L-Achse ist die Helligkeit, die j-Achse ist die Farbtonkomplementarität von Blau und Gelb, 
wobei positive Werte gelber sind und negative Werte blauer sind.
Die g-Achse ist die Farbtonkomplementarität von Rot und Grün, 
wobei positive Werte grüner sind und negative Werte sind pinker ~\cite{OSA-UCS_2023}.

\paragraph{IPT}
Der IPT Farbraum ist ein dreidimensionaler Farbraum.
I steht für die Intensität und P und T ähneln a* und b* von CLAB. 
P beschreibt die Rot-Grün Dimension und T beschreibt die Blau-Gelb Dimension.
Die Vorteile von IPT sind, dass es eine bessere Repräsentation von hue 
in Anlehnung an die menschliche Wahrnehmung bietet. 
Außerdem ist IPT sehr simpel Konzipiert und lässt sich sehr einfach aus dem CIEXYZ Farbraum berechnen ~\cite{Ebner_1998}.

\section{Der Oklab Farbraum}
Das korrekte Darstellen von Farben durch Software ist heutzutage kein Problem mehr, 
anders sieht es bei der der Manipulation von Farben aus.
Typische Farboperationen in Software sind unter anderem das realistische rendern von Bildern, 
das stilisieren und bearbeiten von Bildern, digitales Malen und das Erstellen von Farbverläufen. 
Um dies zu ermöglichen können drei unterschiedliche Vorgehensweisen verwendet werden ~\cite{Ottosson_2020}.
\begin{itemize}
  \item \textbf{Nacharmen des Verhaltens von echtem Licht und Materialien.} Beispielsweise das hinzufügen von Lens Blur, das Hinzufügen von realistischem Nebel oder Raytracing.
  \item \textbf{Nacharmen der menschlichen Farbwahrnehmung,} mit dem Ziel eine intuitive und nachvollziehbare Farbmanipulation zu ermöglichen. Zum Beispiel für das Anpassen der des Chroma ohne die Farbe oder Helligkeit zu verändern.
  \item \textbf{Stilisierte Operationen ohne Realitätsbezug.} 
\end{itemize}

\subsection{Farbverarbeitung heute}
Da Farbe sowohl eine physikalische als auch psychologische Komponente hat, könnte man nun denken, 
dass Farbverarbeitung in Farbräumen auf Basis der physikalischen Interaktion in der echten Welt oder 
auf Basis der menschlichen Wahrnehmeung erfolgen sollte.
Beides ist aber meistens nicht der Fall, da die meisten Programme der einfachheit halber den weit verbreiteten sRGB Farbraumsystem verwenden.
Auch Bildbearbeitungsprogramme wie Photoshop nutzen meistens einfach denselben Farbraum, 
in dem die Bilder gespeichert werden, um Manipulationen durchzuführen. 
Zum einen werden so meist Farbräume verwendet, die für die Farbreproduktion entwickelt wurden, 
statt welche für das durchführen von Berechnungen.
Zum anderen werden immer die Farbräume verwendet, in denen das Bild gespeichert ist, 
was zu inkonsistenten Ergebnissen zwischen verschiedenen Bildformaten führen kann.
Dies sorgt dafür, dass Farbverarbeitung zu einem nebenprodukt der Defonition des Farbraums wird und 
mit sRGB als verbreiteter Standard ist die definiert danach, wie CRT displays in den 80ern Funktioniert haben.
Um dieses Problem zu beheben wurden über die Jahre viele Funktionen in den Bildbearbeitungsprogrammen implementiert, 
diese sollten das Verhalten an die menschliche Wahrnehmung oder das physikalische Verhalten von Licht anpassen. 
Dennoch wurde der zugrundeliegende Fehler eines nicht zweckdienlichen Farbraumsystems nicht behoben ~\cite{Ottosson_2020}.

Um die Probleme mit der Bildbearbeitung in nicht geeigneten Farbräumen zu visualisieren folgen nun ein paar Beispiele. 
In diesen Beispielen werden drei verschiedene Farbverläufe dargestellt, 
der erste basiert auf einem Farbraum der an die menschliche Wahrnehmung angepasst ist, 
der zweite ist ein linearer Farbraum und veranschaulicht gut die physikalische Mischung von Licht, 
der dritte ist ein Farbverlauf in sRGB.
Hier können nun einige unterschiede zwischen den Farbverläufen beobachtet werden.
\begin{itemize}
  \item Weder sRGB noch der lineare Farbraum sind in der Lage Farben mit Weiß zu mischen, ohne dass der eindruck eines konstanten Hue aufrechterhalten wird. Besonders gut erkennbar ist dies bei dem blauen Farbverlauf in Grafik \ref{fig:vergleich_white}.
  \item sRGB dunkelt Farben stark ab, wenn unterschiedliche Farben mit hoher Sättigung gemischt werden. Dies ist besonders gut bei dem Farbverlauf von Grün zu Magenta in Grafik \ref{fig:vergleich_zweifarbig} zu erkennen.
  \item Sowohl sRGB als auch der wahrnehmungsbasierte Farbraum mischen Schwarz und Weiß sehr ähnlich (siehe Grafik \ref{fig:vergleich_white}), während der lineare Farbraum viel heller erscheint. Dies ist der Grund weshalb sRGB manchmal auch als wahrnehmungsbasierter Farbraum bezeichnet wird, trotz all seiner Probleme.
\end{itemize}
In dem Vergleich ist zu erkennen des sRGB weder die menschliche Farbwahrnehmung 
noch das physikalische Verhalten von Licht korrekt abbildet.
Daher ist sRGB nicht für die Farbmanipulation von Bildern geeignet ~\cite{Ottosson_2020}.

\begin{figure}
  \centering
  \includegraphics[width=0.29\textwidth]{Grafiken/Farbverlauf/whiteblue.png}
  \includegraphics[width=0.29\textwidth]{Grafiken/Farbverlauf/whitegreen.png}
  \includegraphics[width=0.29\textwidth]{Grafiken/Farbverlauf/whiteblack.png}
  \caption{Drei Farbverläufe von Weiß zu Blau, Grün und Schwarz. Quelle: ~\cite{Ottosson_2020}}
  \label{fig:vergleich_white}
\end{figure}

\begin{figure}
  \centering
  \includegraphics[width=0.49\textwidth]{Grafiken/Farbverlauf/blueyellow.png}
  \includegraphics[width=0.49\textwidth]{Grafiken/Farbverlauf/greenmagenta.png}
  \caption{Zwei Farbverläufe, links von Blau nach Gelb, rechts von Grün nach Magenta. Quelle: ~\cite{Ottosson_2020}}
  \label{fig:vergleich_zweifarbig}
\end{figure}

\subsection{Motivation für Oklab}
Da nun klar ist, dass die häufig verwendeten Farbräume nicht für die Farbmanipulation geeignet sind, 
muss ein neuer Farbraum entwickelt werden, der diese Aufgabe erfüllt.
%Ziele oklab

\paragraph{Helmholtz-Kohlrausch Effekt}


\subsection{Herleitung von Oklab}
\subsection{Wie gut ist Oklab?}
\subsection{Bessere Sättigungskorrektur mittels Oklab}
% Von dem anderen Artikel
\subsection{Oklab in der Praxis}

% Bibliography
\printbibliography

\end{document}
